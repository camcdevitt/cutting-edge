\textbf{Summary}

Metal ions are essential to all forms of life. Bacterial pathogens scavenge transition metal ions from the host environment by use of high (nanomolar) affinity permeases. How these transporters discriminate and acquire their specific metal ion cargo has remained unknown. Here we showed that the manganese-specific Psa permease was not selective for metal ion interaction, but instead achieved functional specificity by only releasing metal ions that favour six-coordinate binding. The concerted combination of structural changes that occur during metal binding results in the partial unwinding of a rigid linking alpha-helix. First-row transition metal ions that employ six-coordinate binding, such as manganese, facilitate a reversible distortion of the linking alpha-helix. Whereas metal ions that prefer four-coordinate binding, such as zinc, result in greater distortion of the helix that prevents release of the bound metal ion. Collectively, these findings highlight how biology has overcome the insurmountable challenge of achieving selectivity for the acquisition of lower order first-row transition elements at the host-pathogen interface.  Further, this work provides new insight into how bacterial metal-receptors scavenge essential metal ions from the host environment and the molecular basis for how zinc can be toxic to bacterial organisms. Our detailed functional and structural understanding of bacterial metal ion receptors provides a foundation for the development of novel antimicrobial compounds. 